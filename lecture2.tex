\section{}

Recall last lecture we analysed $\dot{x} = \lambda - \alpha x V x_\tau$ by
\begin{enumerate}
  \item finding a steady state solution $x=x^*2$
  \item linearalizing in the neighbourhood of the solution
  \item changing variables to $z = x- x^*$
  \item using the eigenvalue equation to find \[\tau =\frac{\cos^{-1} (-A/B)} {\sqrt{B^2-A^2}}\]
\end{enumerate}

Note that we must have 
$\frac{A}{B}\le1$. Additionally, since $\omega=\sqrt{B^2-A^2}$, we can show that \[T\ge2\tau, \arccos(-A/B)\in [0,\pi]\] and \[2\tau\le T\le 4\tau, \arccos(-A/B)\in [\pi/2, \pi]\]

%graph (-A/B) vs arccos(-A/B)

\begin{align*}
  \tau &= \frac{cos(-A/B)}{B\sqrt(1-(A/B)^2)}\\
\end{align*}

Note that since $x\sim mmHg, v\sim L/min, S\sim L/(min\cdot mmHg)$, $A/B$ is dimensionless.

Normal values for the parameters are
\begin{align*}
  \tau\sim 0.25min\\
  x^*\sim 40mmHg\\
  V^*\sim 7L/min\\
  \lambda\sim 6 mmHg/min\\
  S^*\sim 4L/(min\cdot mmHg)
\end{align*}

$A/B = 7/(40\cdot 4) =7/160 \approx 0.04$

Since $A/B$ is small, $\arccos{-A/B} \approx \pi/2 \Rightarrow \omega\tau = \arccos(-A/B)\approx \pi/2$.

%Also, $B\approx $

$\tau \approx \pi/(2\cdot 3.4)\approx 0.46 minutes$

$\tau = \frac{cos(-A/B)}{B\sqrt(1-(A/B)^2)}$

Assertion: If $\tau < \tau_c$, then $x^*$ is locally stable. Physiological $\tau=0.25min$, calculated $\tau_c = 0.46 min$.

Approximately: $\omega\tau_c = \pi/2 = B\tau_c = \lambda S^*/V^*\tau_c$

Should have local stability if $\tau \le \pi/2 \cdot V^*/(\lambda S^*)$.

\vbox{}

recall $\nu = - A - B e^{-\nu\tau_c}$. Suppose we make perturbations in the form 

\begin{align*}A
  \rightarrow A+a\\B\rightarrow B+b\\ \tau_c\rightarrow \tau_c + \epsilon
\end{align*}

This expands (since $\epsilon\Delta\nu\in O(\epsilon^2)$) to:
\[ \nu + \Delta\nu \approx -(A+A) - (B+b)e^{-\nu\tau_c} + (B+b)(\tau_c \Delta \nu +\nu\epsilon)e^{-\nu\tau_c}\]

So $\Delta\nu\approx -a-be^{-\nu\tau_c}+ Be{-\nu\tau_c}(\tau_c\Delta\nu+\epsilon\nu)$

Assuming only changes in $\tau \Rightarrow a=b=0$.

$\Delta\nu\approx Be^{-\nu\tau_c}(\tau_c\Delta\nu+\epsilon\nu)$

%$1-B\tau_c e^{-\nu\tau_c}\Delta\nu\approx \epsilon\nu\Be^{-\nu\tau_c}$

Recall $\nu = i\omega$. Then
\begin{align*}
  Re[\Delta\nu] &= -\epsilon/D [-\omega^2(A\tau_c)+A\omega^2\tau_c]\\
  &= \epsilon\omega^2/D
\end{align*}

%Where D is a denominator\dots

Assumed small change in 3 parameters. Then solved for change in nu given small changes, simplified by assuming change in one variable. Attempt to answer question: if $\tau\rightarrow\tau+\epsilon,\quad\epsilon>0$, what happens to the real part of nu? We started with a solution where nu was purely imaginary, and had the requirement that it develop a real solution when tau becomes less than the critical value. We showed this\dots apparently.
